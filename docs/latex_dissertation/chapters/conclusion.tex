\chapter{Conclusion}
\label{sec:concl}

To complete the dissertation a summary of the results and answers obtained regarding the research questions. Thereafter, limitations and an outlook on supplementary options to the approaches presented in this work are provided.

\section{Summary and Implications}

In this work, a deterministic robust cost-minimising unidirectional day-ahead scheduling routine for charging electric vehicles overnight in residential low voltage distribution networks has been presented, that observes local network, equipment and charging demand constraints in a stochastic environment. The stochastic environment involves uncertain residential electricity demand, market prices, and the mobility behaviour of electric vehicle owners including stochastic daily trip distances, arrival and departure times. Knowledge about the probability distributions of these parameters is used to hedge risks regarding the costs of charging, network overloadings and voltage violations, and vehicle demand fulfilment.

In \Autoref{sec:rq} the goals and research questions of this work have been set.

\textbf{Research Question 1} asked about the characterisation and modelling of the major uncertainties involved in EV scheduling, which were derived in \Autoref{sec:model}. Residential demand indefiniteness, considered at each individual household, is represented by a random shift of a forecasted load profile within a set window assuming historical data on characteristic demand peaks. Spot market price uncertainty is modelled by a normally distributed red noise error sequence to reflect the serial correlation of prediction errors, that is magnified for price extremes. From extensive survey data, a mobility model based on few continuous distributions is built. Its consideration of not only scenario uncertainty about average travel behaviour of households with EVs but also a typical deviation from this average is a prime advantage in accuracy.

\textbf{Research Question 2} inquired the robustness of analytical and heuristic scheduling methods under uncertainty. To circumvent the complexity induced by nonlinear power flow and receive solutions quickly from linear programming, a method for linear power flow approximation based on network sensitivities is introduced. Besides remarkably accurate results, corresponding analysis reiterated the predisposition of households further along the feeder to be more sensitive to additional loads in the network. While reference optimisations underlined the poor performance of uncoordinated charging, linear programming could only marginally outrun the heuristic scheduling methods regarding charging costs but even succumbs in terms of technical constraint violation severity. Due to uncertainty, a full battery could not be guaranteed, and while average charge levels are at 95\%, occasionally charge levels amount to mere 50\%. The fact that the cost increase from purely price-based to network-constrained optimisation is confined to the order of 10\% on average is remarkable and indicates that numerous alternative slots with similar charging costs and suitable capacities are available to avoid constraint violations.

\textbf{Research Question 3} further regarded the performance trade-off by uncertainty mitigation options concerning charging cost and technical intactness. Compared to previous approaches addressing the stochasticity in EV scheduling, the simplicity of presented approaches protrudes. The same deterministic optimisation method is employed and adapts to varying levels of uncertainty by entering more conservative estimates of the probabilistic input parameters. The attenuation of residential demand uncertainty eliminated overloads and voltage violations at negligible extra cost. The consideration of market price uncertainty by damping cheapest but most variable slots was effective at confining the range of realised charging costs. Tackling availability uncertainty had only a limited effect since slow ramping of charging rates in the linear programme already mitigates inherently by not allocation maximum charge rates in rather uncertain boundary regions of predicted availability. The trade-off between cost and reliability became most evident when addressing battery charge level uncertainty.  With increasing security margins, while the reliability of EV demand satisfaction rises and especially the minimum final battery charge increases rapidly, the charging costs increase since vehicles are scheduled to provide more energy than they are expected to require.

\newpage
\textbf{Research Question 4} concerned the influence of electricity tariff designs on cost saving potential. Low, medium and high price spreading factors are tested, and it is shown that the magnitude of savings depends on the spreading of the price signal, ranging from 10\% to more than 40\%. The availability of high price spreads raised cost savings on average significantly and subordinately their daily indefiniteness. Therefore, dynamic charges, which amplify the effect of variable electricity prices, constitute a driver to incentivise demand-side management with electric vehicles further and, thereby, accelerate the transition towards a more sustainable transport and electricity sector with little fiscal implication.

\section{Limitations and Outlook}

Although interesting insights into the deterministic uncertainty mitigation of electricity prices, mobility behaviour and residential demand in EV scheduling could be provided and clear improvements compared to naive reference optimisation approaches for both financial and technical domains could be shown, there are several more promising aspects deserving consideration in the future which were not elaborated in this work as they would exceed its scope due to the limited editing time.

In this work, electric vehicles are considered to be the only deferrable load at households. In the future, it could be examined how optimisation of a pool of controllable devices can be conducted taking into account their interdependencies on a home level. This especially gains importance due to the advent of renewable heat applications, namely heat pumps and combined heat and power (CHP) devices, adding further loads to the network but also energy storage capabilities \cite{Neumann2016}. Furthermore, the interaction with local photovoltaic generation bears the potential to minimise power exchange with the distribution system and, consequently, by increasing self-reliance to contain detrimental grid impacts of both distributed generation and electric vehicle loads if controlled judiciously \cite{Allerding2014}. A challenge that has to be overcome is the temporal mismatch of PV generation and residential EV availability. In this context, it is also conceivable to transpose the optimisation routine from residential areas to more industrial regions where electric vehicles charge in the parking lots of office buildings with similar degrees of load flexibility. This might even expose less uncertainty due to more reliable working times and locational aggregation of loads.

Moreover, on an aggregate level, it would be of further value to assess the impact of many coordinating aggregators after a significant market uptake of electric vehicles on electricity markets and transmission system operation, particularly, whether the assumption that aggregators act as price takers on the market and do not exert market power holds. Commonly the influence of additional EV loads on electricity prices is disregarded. Additionally, research on the optimal sizing of scheduling problems and more decentralised approaches is pivotal, as centralised decision making will become impractical and intractable considering its rising computational complexity \cite{Mukherjee2015}.

The dissertation further focusses on day-ahead scheduling with minimal real-time adaptation. To find out to what extend a rolling optimisation horizon can increase cost savings and robustness towards uncertainties due to a more precise updated predictions, uncertainty has to be introduced to the simulation environment in a way such that prediction quality diminishes the further it diverges from the starting point of the optimisation horizon. Moreover, an intelligent controller could be superimposed on the optimised schedule to control charging processes at a higher resolution than 15 minutes and address neglected volatilities as well as deviations from predicted parameters.

Moreover, the presented concepts for uncertainty mitigation primarily rely on the assumption that based on the empirical data of a user's behaviour, probabilities for input parameters can be defined similarly to the modelled continuous distribution functions. This raises two issues: First, the consideration of mobility behaviour uncertainty demands a certain lead time or more rigorous manual communication with the EV owner to collect data on typical behaviour. Second, once required data is available, it must be exchanged with the aggregator in consideration of data privacy concerns. In the context of distribution assumptions, it must further be highlighted that the mitigation of price uncertainties relies on the assumption that forecasts can be given with certain error margins and distributions.

Ancillary service provision has only been marginally touched on by the ability to provide negative regulation capacity. Activation of this capacity or even bi-directional power flow by discharging batteries would add a greater degree of uncertainty to the scheduling process and require accurate modelling of the respective markets, which tend to be more complex and less uniform across different countries \cite{Ocker2016,Siddiqui2001}. While technical and economic aspects have been addressed separately, the impact of market participation on security and reliability in the distribution network is a non-concluded research question. Furthermore, a valuation of installing bi-directional charging infrastructure against the background of different subsets of regulation service markets could be of value for future research.

It is also assumed that intelligent charging control is always financially preferred to network reinforcement. However, limited network reinforcement at bottlenecks of the distribution network might enhance the relative advantage of smart EV scheduling \cite{Papadopoulos2012}. Consequently, a long-term trade-off analysis to determine the optimal mix of network equipment upgrades and charging coordination to minimise infrastructure and operating costs could shape future research projects.

Future tariff-design by amplifying wholesale market signals is a promising lead which still requires further examination from the policy and operational perspective.

In conclusion, although several requirements apply for the well-functioning of the proposed uncertainty mitigation approaches in reality due to the simplifying modelling assumptions, they are not unlikely to be fulfilled. Further research will extend the joint optimisation and scheduling of a variety of distributed resources, market power of aggregators, future tariff-design, the economic feasibility of ancillary service provision and the consideration of minor network reinforcement for an additional economic benefit through coordinated EV charging.